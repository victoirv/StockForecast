Basic framework for using a variety of models to forecast and simulate stock market performance.





\subsection*{To-\/\+Do}


\begin{DoxyItemize}
\item Switch getopt to argparse
\item Command line arguments for start and end dates
\item Neural network model
\item Add more performance metrics
\begin{DoxyItemize}
\item Good/\+Total predictions
\item Performance vs random
\item Portfolio improvement as function of stock improvement
\end{DoxyItemize}
\item Improve optimization code to avoid overfitting
\item Add makefile for doxygen, include profiling graph (code here until then)
\begin{DoxyItemize}
\item python -\/m c\+Profile -\/o logfile \hyperlink{stocks_8py}{stocks.\+py}
\item gprof2dot -\/f pstats logfile -\/o dotfile
\item dot dotfile -\/\+Tpng pngfile
\end{DoxyItemize}
\item Figure out why S\&P and D\+J\+IA seemingly observe different holidays
\item Make documentation more doxygen friendly
\end{DoxyItemize}

\subsection*{\href{docs/html/index.html}{\tt Doxygen documentation}}





\subsection*{History}


\begin{DoxyItemize}
\item \href{#april-4-2017}{\tt April 4\+: Initial Commit}
\end{DoxyItemize}

\subsubsection*{April 4, 2017}

Making the initial commit of this code to github. As it stands, the functionality ported over from my matlab stock framework is limited to linear and random models, but overall generalizability and functionality should be greatly expanded by using python. This also allows for installation on a server without another matlab license, which is nice.

Sample of the current state of predictions\+:

\tabulinesep=1mm
\begin{longtabu} spread 0pt [c]{*{2}{|X[-1]}|}
\hline
\rowcolor{\tableheadbgcolor}\PBS\centering {\bf \hyperlink{namespaceIR}{IR} Model }&\PBS\centering {\bf Randombuy Model  }\\\cline{1-2}
\endfirsthead
\hline
\endfoot
\hline
\rowcolor{\tableheadbgcolor}\PBS\centering {\bf \hyperlink{namespaceIR}{IR} Model }&\PBS\centering {\bf Randombuy Model  }\\\cline{1-2}
\endhead
\PBS\centering  &\PBS\centering  \\\cline{1-2}
\end{longtabu}
Working well at buying low and selling high short term, but definitely needs improvement in forecasting long term trends. Performance isn\textquotesingle{}t great for stocks in continual decline (which makes sense, except previous nonlinear models have set high expectations for being able to avoid this problem).

Also, just for fun, the profiler graph of a run\+:  